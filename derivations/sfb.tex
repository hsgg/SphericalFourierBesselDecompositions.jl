\documentclass[aps,prd,reprint,floatfix,superscriptaddress,showkeys,nofootinbib]{revtex4-1}




\bibliographystyle{apsrev4-2-truncate}
\setcitestyle{authoryear,round}
\setlength{\bibsep}{4pt}

\usepackage[utf8]{inputenc}
\usepackage[T1]{fontenc}


\usepackage{graphicx}

\usepackage{savesym}
\savesymbol{tablenum}
\usepackage{siunitx}
\sisetup{group-separator={,},group-digits=integer}
\restoresymbol{SIX}{tablenum}

\usepackage{mathtools}
\savesymbol{corresponds}
\usepackage{mathabx}  % for \widebar
\restoresymbol{ABX}{corresponds}
\usepackage{bm}

\usepackage{verbatim}
\usepackage{rotate}
\usepackage{color}
\usepackage{aas_macros}
\usepackage{tikz}
\usetikzlibrary{calc}
\usepackage{ulem}
\normalem  % https://tex.stackexchange.com/questions/104058/getting-emph-back-to-normal-after-loading-ulem


\DeclareFontFamily{OT1}{pzc}{}
\DeclareFontShape{OT1}{pzc}{m}{it}%
            {<-> s * [1.10] pzcmi7t}{}
\DeclareMathAlphabet{\mathscr}{OT1}{pzc}%
                                {m}{it}

\definecolor{RedWine}{rgb}{0.743,0,0}
\definecolor{green(pigment)}{rgb}{0.0,0.65,0.31}
\definecolor{RoyalBlue}{rgb}{0.25,0.41,0.88}
\newcommand{\hg}[1]{\textcolor{green(pigment)}{\textbf{(HG: #1)}}}
\newcommand{\Oli}[1]{\textcolor{RedWine}{\textbf{(Oli: #1)}}}

\newcommand{\be}{\begin{equation}}
\newcommand{\ee}{\end{equation}}
\newcommand{\bea}{\begin{eqnarray}}
\newcommand{\eea}{\end{eqnarray}}
\def\ba#1\ea{\begin{align}#1\end{align}}

\def\({\left(}
\def\){\right)}
\def\<{\left\langle}
\def\>{\right\rangle}


\newcommand{\vs}{\nonumber\\} 

\def\vK{{\bm{K}}}
\def\vR{{\bm{R}}}
\def\vr{{\bm{r}}}
\def\vx{{\bm{x}}}
\def\vv{{\bm{v}}}
\def\vp{{\bm{p}}}
\def\vvs{{\bm{s}}}
\def\vk{{\bm{k}}}
\def\vY{{\bm{Y}}}
\def\vy{{\bm{y}}}
\def\vq{{\bm{q}}}
\def\vu{{\bm{u}}}
\def\vz{{\bm{z}}}
\def\vgam{{\bm{\gamma}}}
% Caught you peaking at the source code!
\def\vPsi{{\bm{\Psi}}}

\def\vecl{\vec{l}}
\def\vecL{\vec{L}}
\def\vecv{\vec{v}}
\def\veck{\vec{k}}
\def\vecp{\vec{p}}
\def\vecx{\vec{x}}
\def\vect{\vec{\theta}}

\def\JM{{(\ell m)}}
\def\lm{{(\ell m)}}
\def\zt{\tilde{z}}
\def\chit{{\tilde\chi}}
\def\bt{b_e}
\def\H{\mathcal{H}}
\def\Gaunt#1#2#3#4#5#6{\mathcal{G}^{#1}_{#4}{}^{#2}_{#5}{}^{#3}_{#6}}
\def\PA{\mathrm{PA}}

\def\ehat{{\hat{\bm{e}}}}
\def\nhat{{\hat{\bm{n}}}}
\def\khat{{\hat{\bm{k}}}}
\def\Khat{{\hat{\bm{K}}}}
\def\xhat{{\hat{\bm{x}}}}
\def\vhat{{\hat{\bm{v}}}}
\def\phat{{\hat{\bm{p}}}}
\def\qhat{{\hat{\bm{q}}}}
\def\rhat{{\hat{\bm{r}}}}
\def\Rhat{{\hat{\bm{R}}}}
\def\shat{{\hat{\bm{s}}}}
\def\yhat{{\hat{\bm{y}}}}
\def\zhat{{\hat{\bm{z}}}}
\def\Nbar{{\bar{N}}}
\def\nbar{{\bar{n}}}

\def\fnl{f_\mathrm{NL}}
\def\erfc{\mathrm{erfc}}
\def\Pp{P_\Phi}
\DeclareMathOperator{\erf}{erf}

\def\orderof{\mathcal{O}}
\newcommand{\half}[1][1]{\ensuremath{\frac{#1}{2}}}
\newcommand{\quarter}[1][1]{\ensuremath{\frac{#1}{4}}}

\DeclareSIUnit \parsec {pc}
\DeclareSIUnit \h {\text{$h$}}
\DeclareSIUnit \year {yr}
\DeclareSIUnit \solarmass {M_\odot}
\DeclareSIUnit \Mpc {\mega\parsec}
\newcommand{\hMpc}{\ensuremath{\mega\parsec\per\h}}

\def\tot{\rm total}
\def\oii{\mathrm{OII}}
\def\oiii{\mathrm{OIII}}
\def\lae{\mathrm{LAE}}
\def\lin{\mathrm{lin}}
\def\hae{\mathrm{HAE}}

\def\th{\mathrm{th}}
\def\obs{\mathrm{obs}}
\def\proj{\mathrm{proj}}
\def\true{\mathrm{true}}
\def\inferred{\mathrm{inferred}}
\def\est{\mathrm{est}}  % current estimate (in an iterative setting)
\def\min{\mathrm{min}}
\def\max{\mathrm{max}}
\def\mid{\mathrm{mid}}
\def\sky{\mathrm{sky}}
\def\fill{\mathrm{fill}}

\def\xlae{x_\lae}
\def\xoii{x_\oii}
\def\Plin{P_{\rm Lin}}

\def\tr{\mathrm{tr}}
\def\atanh{\mathrm{atanh}}

\def\Like{\mathfrak{L}}
\def\f{\mathfrak{f}}
\def\g{\mathfrak{g}}
\def\ffid{\f_\mathrm{true}}
\def\gfid{\g_\mathrm{true}}
\def\Q{{\cal Q}}

\def\dd{\mathrm{d}}
\def\deriv#1{\mathrm{\romannumeral #1}}

\def\myapp#1#2{%
  \mathrel{%
    \setbox0=\hbox{$#1\sim$}%
    \setbox2=\hbox{%
      \rlap{\hbox{$#1\propto$}}%
      \lower1.1\ht0\box0%
    }%
    \raise0.25\ht2\box2%
  }%
}
\def\approxpropto{\mathpalette\myapp\relax}


\newcommand{\incgraph}[2][0.49]{\includegraphics[width=#1\textwidth]{#2}}

\newcommand{\changed}[1]{\textcolor{red}{#1}}
\newcommand{\changedd}[1]{#1}%\textcolor{black}{#1}}

\usepackage[colorlinks]{hyperref}
\usepackage[capitalise]{cleveref}
\newcommand{\crefrangeconjunction}{--}

\hypersetup{
colorlinks=true,
citecolor=cyan,
pagebackref=true,}


\begin{document}

\title{SuperFaB: a fabulous code for Spherical Fourier-Bessel decomposition}

\author{Henry S. \surname{Grasshorn Gebhardt}}
\email{henry.s.gebhardt@jpl.nasa.gov}
\thanks{NASA Postdoctoral Program Fellow}
\affiliation{Jet Propulsion Laboratory, California Institute of Technology, Pasadena, CA 91109, USA}
\affiliation{California Institute of Technology, Pasadena, CA 91125, USA}

\author{Olivier \surname{Dor\'e}}
\affiliation{Jet Propulsion Laboratory, California Institute of Technology, Pasadena, CA 91109, USA}
\affiliation{California Institute of Technology, Pasadena, CA 91125, USA}




\begin{abstract}
	Here we derive supplemental things for the code.
\end{abstract}

\keywords{cosmology; large-scale structure}


\maketitle

\newcommand{\codename}{\emph{SuperFaB}}





\section{Window and selection function}
The observed number density $n(\vr)$ of galaxies is subject to the window and
selection function $W(\vr)$ of the survey, which we define as the fraction of
galaxies observed at position $\vr$. For a random catalogue subject to the same
window function, with density $n_r(\vr)$, and with $1/\alpha$ as many galaxies
as the survey, we then have
\ba
\label{eq:window_definition}
\alpha\<n_r(\vr)\> = W(\vr) \, \bar n\,,
\ea
where $\<n_r(\vr)\>=\alpha^{-1}\bar n(\vr)$ is the average number density of
the ensemble of random catalogs, and $\bar n$ is the average number density in
the survey. Note that \cref{eq:window_definition} can equivalently be expressed
in terms of the limit $\lim_{\alpha\to0}\alpha\,n_r(\vr)=\bar
n(\vr)=W(\vr)\,\bar n$. With this definition of the window function, we define
the effective volume as
\ba
\label{eq:Veff}
  V_\mathrm{eff} &= \int\dd^3\vr\,W(\vr)\,,
\ea
so that the average number density $\bar n$ becomes
\ba
\label{eq:nbar}
  \bar n &= \frac{N_\mathrm{gal}^\obs}{V_\mathrm{eff}}\,,
\ea
and $N_\mathrm{gal}^\obs$ is the observed number of galaxies in the survey. Any
variation across the survey in the actual average number density, e.g., due to
an evolving luminosity function, is absorbed into $W(\vr)$.
Our treatment is in line with \citet{Taruya+:2021PhRvD.103b3501T}, and our
$W(\vr)$ takes the role of the function $G(\vr)$ in
\citet{Feldman+:1994ApJ...426...23F}, except that we do not at present include
a weighting scheme.

In a sense, there are two window functions here: first, the one defined by the
SFB procedure and limited by $r_\min \leq r \leq r_\max$, and second, $W(\vr)$,
which defines the geometry and selection of the survey. However, the first one
should be irrelevant as long as the survey volume is entirely inside $r_\min
\leq r \leq r_\max$ and as along as sufficient number of modes are included in
the SFB analysis.

The observed density fluctuation field is, then,
\ba
  \delta^\obs(\vr)
    &= \frac{n(\vr) - \alpha\,n_r(\vr)}{\nbar}
    = \frac{n(\vr)}{\nbar} - W(\vr)\,,
  \label{eq:density_constrast_discrete_points}
\ea
where \cref{eq:window_definition} was used in the limit that the random
catalogue has an infinite number of galaxies, or $\alpha\to0$. Because the
observed density $n(\vr)$ is also subject to the window function $W(\vr)$, the
observed and true density contrasts are related by
\ba
\delta^\obs(\vr) &= W(\vr)\, \delta^A(\vr)\,,
\ea
where we attach the superscript `A' to refer to the local average effect (see
\cref{sec:sfb_local_average_effect} below).
Transforming to SFB-space and expressing $\delta^A(\vr)$ in terms of its SFB
decomposition
\cref{eq:sfb_discrete_fourier_pair_a,eq:sfb_discrete_fourier_pair_b}, we get
\ba
\delta^\obs_{n \ell m}
&=
\sum_{n'\ell'm'}
W_{n \ell m}^{n'\ell'm'}
\,\delta^A_{n'\ell'm'}\,,
\label{eq:delta_mixing}
\ea
where
\ba
\label{eq:delta_mixing_matrix_discrete}
W_{n \ell m}^{n'\ell'm'}
&=
\int\dd r\,r^2
\,g_{n\ell}(r)
\,g_{n'\ell'}(r)
\vs&\quad\times
\int\dd^2\rhat
\,Y^*_{\ell m}(\rhat)
\,Y_{\ell'm'}(\rhat)
\,W(r,\rhat)\,.
\ea

\subsubsection{Properties and implementation}
From \cref{eq:delta_mixing_matrix_discrete} follows the symmetry
\ba
\label{eq:sfb_window_parity}
W_{n,\ell,m}^{n',\ell',-m'}
&=
(-1)^{m+m'}
W_{n,\ell,-m}^{n',\ell',m',*}
\,,
\ea
and the Hermitian property
\ba
\label{eq:sfb_window_symmetry}
W_{n\ell m}^{n' \ell' m'}
&=
W^{n\ell m,*}_{n' \ell' m'}\,.
\ea
In the special case that $W(\vr)=1$ everywhere,
\ba
W_{n \ell m}^{n'\ell'm'}
&=
\delta^K_{nn'}
\delta^K_{\ell\ell'}
\delta^K_{mm'}\,,
\ea
which follows from \cref{eq:gnl_orthonormality} and \cref{eq:YlmYlmDelta}.

In all generality, \cref{eq:delta_mixing_matrix_discrete} can be simplified for computational
convenience by expressing the window function in terms of an angular
transform. That is, introduce
\ba
\label{eq:win_r_lm}
W_{LM}(r)
&=
\int\dd^2\rhat\,Y^*_{LM}(\rhat)\,W(r,\rhat)\,.
\ea
Then,
\ba
W_{n \ell m}^{n'\ell'm'}
&=
(-1)^m
\sum_{L}
\mathcal{G}^{\ell\ell'L}_{-m,m',m-m'}
\vs&\quad\times
\int\dd r\,r^2
\,g_{n\ell}(r)
\,g_{n'\ell'}(r)
\,W_{L,m-m'}(r)
\label{eq:delta_mixing_matrix_discrete_sht}
\,, 
\ea
where we used \cref{eq:Ylm_conjugate} and introduced the Gaunt
factor \cref{eq:gaunt_factor}.
In writing \cref{eq:delta_mixing_matrix_discrete_sht} we performed the angular
transform of the window function only as that leads to a computationally
suitable form. Had we performed a full SFB transform, we would have been left
with an infinite sum over $n$ that converges slowly, in addition to the need of
computing integrals over three spherical Bessel functions.



\begin{widetext}
\section{Power spectrum estimation}
\citet{Wandelt+:2001PhRvD..64h3003W,Hivon+:2002ApJ...567....2H} use a
\emph{pseudo-$C_\ell$} method to estimate the power spectrum. Translating to
the SFB decomposition, the pseudo-$C_\ell$ method assumes that much of the
information about the power spectrum is contained in the pseudo-power spectrum
\ba
\label{eq:sfb_clnnobs}
\hat C^{XY,\obs}_{\ell nn'}
   &= \frac{1}{2\ell+1}\sum_m \delta^{X,\obs}_{n\ell m} \delta^{Y,\obs,*}_{n'\ell m}\,.
\ea
That is, we ignore off-diagonal terms $L\neq\ell$ and $M\neq m$, and average
over $m$. The effect of the window is then described by a mixing matrix between
the $\hat C^\obs_{\ell nn'}$ and $\hat C^A_{\ell nn'}$,
\ba
\label{eq:cell_mixing_matrix}
C^{XY,\obs}_{\ell nn'}
   &=
   \frac{1}{2\ell+1}\sum_m
   \sum_{NLM} \sum_{N'L'M'}
   W_{n\ell m}^{X,NLM}
   W_{n'\ell m}^{Y,N'L'M',*}
   \<\delta^{X}_{NLM} \delta^{Y,*}_{N'L'M'}\>
   \\
   &= \sum_{LNN'} \frac{1}{2\ell+1}\sum_{mM}
   W_{n\ell m}^{X,NLM}
   W_{n'\ell m}^{Y,N'LM,*}
   C^{XY,A}_{LNN'}
   \\
   &= \sum_{LNN'} \mathcal{M}_{\ell nn'}^{XY,LNN'}\,C^{XY,A}_{LNN'}
   \,,
\ea
where we used \cref{eq:delta_mixing} and defined
\ba
\label{eq:sfb_cmix}
\mathcal{M}_{\ell nn'}^{XY,LNN'}
&=
\frac{1}{2\ell+1}\sum_{mM}
W_{n\ell m}^{X,NLM}
W_{n'\ell m}^{Y,N'LM,*}
\,,
\ea
and the index `A' on $C^A_{\ell nn'}$ indicates the local average effect, see
\cref{sec:sfb_local_average_effect}. Next, with
\cref{eq:win_r_lm,eq:delta_mixing_matrix_discrete_sht} we get
\ba
\mathcal{M}_{\ell nn'}^{XY,LNN'}
&=
\frac{1}{2\ell+1}\sum_{mM}
\vs&\quad\times
(-1)^m
\sum_{L_1 M_1}
\mathcal{G}^{\ell L L_1}_{-m,M,M_1}
\int\dd r\,r^2
\,g_{n\ell}(r)
\,g_{NL}(r)
\,W^X_{L_1M_1}(r)
\vs&\quad\times
(-1)^m
\sum_{L_2 M_2}
\mathcal{G}^{\ell L L_2}_{-m,M,M_2}
\int\dd r\,r^2
\,g_{n'\ell}(r)
\,g_{N'L}(r)
\,W^{Y,*}_{L_2M_2}(r)
\\
&=
\sum_{L_1 M_1}
\sum_{L_2 M_2}
\frac{1}{2\ell+1}\sum_{mM}
\mathcal{G}^{\ell L L_1}_{-m,M,M_1}
\mathcal{G}^{\ell L L_2}_{-m,M,M_2}
\vs&\quad\times
\int\dd r\,r^2 \,g_{n\ell}(r) \,g_{NL}(r) \,W^X_{L_1M_1}(r)
\int\dd r\,r^2 \,g_{n'\ell}(r) \,g_{N'L}(r) \,W^{Y,*}_{L_2M_2}(r)
\,.
\ea
\end{widetext}
Using the orthogonality of the Gaunt factor \cref{eq:gaunt_orthogonality},
\ba
\mathcal{M}_{\ell nn'}^{XY,LNN'}
&=
\frac{2L+1}{4\pi}
\sum_{L_1}
\begin{pmatrix}
  \ell & L & L_1 \\
  0 & 0 & 0
\end{pmatrix}^2
\sum_{M_1}
\vs&\quad\times
\int\dd r\,r^2 \,g_{n\ell}(r) \,g_{NL}(r) \,W^X_{L_1M_1}(r)
\vs&\quad\times
\int\dd r\,r^2 \,g_{n'\ell}(r) \,g_{N'L}(r) \,W^{Y,*}_{L_1M_1}(r)
\label{eq:cell_mixing_matrix_explicit}
\,.
\ea




\ba
\mathcal{M}_{\ell nn'}^{LNN'}
&=
\frac{2L+1}{4\pi}
\sum_{L_1}
\begin{pmatrix}
  \ell & L & L_1 \\
  0 & 0 & 0
\end{pmatrix}^2
\sum_{M_1}
\vs&\quad\times
\int\dd r\,r^2
\,g_{n\ell}(r)
\,g_{NL}(r)
\,W_{L_1M_1}(r)
\vs&\quad\times
\int\dd r'\,r'^2
\,g_{n'\ell}(r')
\,g_{N'L}(r')
\,W^*_{L_1M_1}(r')
\label{eq:cell_mixing_matrix_explicit}
\,,
\ea
and we used the orthogonality of the Gaunt factor
\cref{eq:gaunt_3j,eq:3j_orthogonality}. (The sum over $M_1$ could be
performed first. However, that approach is much more memory intensive, so that
computing the integrals first ends up being faster. We have also avoided
expressing the result in terms of a full SFB transform, as that would require a
slowly-converging sum over $n$.)
Note that the matrix $(2L+1)^{-1}\mathcal{M}_{\ell nn'}^{LNN'}$ is symmetric under exchange of the set of indices $(LNN')$ and $(\ell nn')$,
but $\mathcal{M}$ by itself is not.


\subsection{Separable mask and radial selection}
It is quite common that the window function is separable into a radial and an
angular term,
\ba
W(\vr) &= \phi(r) \, M(\rhat)\,.
\ea
If the flux limit in a blind
survey is near $L^*$, then the selection could change dramatically as a
function of angular depth variations that are due to, e.g., atmospheric
variations, and the separation of angular and radial selection would be a poor
approximation.
However, eBOSS, for example, had more targets selected in regions where two or
more plates overlapped \citep[e.g.][]{deMattia+:2021MNRAS.501.5616D}.
Similarly, PFS will have higher target numbers where pointings overlap
\citep{Sunayama+:2020JCAP...06..057S}.

When the window function is separable, then \cref{eq:win_r_lm} is separable as
well,
\ba
W_{LM}(r)
&=
\phi(r) \, W_{LM}\,,
\ea
where
\ba
\label{eq:sfb_Wlm}
W_{LM} &= \int\dd^2\rhat\,Y^*_{LM}(\rhat)\,M(\rhat)\,,
\ea
and \cref{eq:cell_mixing_matrix_explicit} becomes
\ba
\mathcal{M}_{\ell nn'}^{LNN'}
&=
\frac{2L+1}{4\pi}
\sum_{L_1}
\begin{pmatrix}
  \ell & L & L_1 \\
  0 & 0 & 0
\end{pmatrix}^2
\sum_{M_1}
\left|W_{L_1M_1}\right|^2
\vs&\quad\times
\int\dd r\,r^2
\,g_{n\ell}(r)
\,g_{NL}(r)
\,\phi(r)
\vs&\quad\times
\int\dd r'\,r'^2
\,g_{n'\ell}(r')
\,g_{N'L}(r')
\,\phi(r')
\,,
\ea
which is also separable, and therefore significantly reduces computation cost.
\cref{eq:delta_mixing_matrix_discrete_sht} simplifies in a similar manner.

In the special case that $W(\vr)=1$ everywhere, we recover the unit matrix
\ba
\mathcal{M}_{\ell nn'}^{LNN'}
&=
\delta^K_{\ell L}
\delta^K_{nN}
\delta^K_{n'N'}
\,,
\ea
as expected.

We give two further examples in \cref{fig:bcmix}. In the left panel, we show
the mixing matrix for a mask covering half the sky, and this leads to coupling
of neighboring $\ell$-modes. On the right, we add a radial selection decreasing
with redshift, and this additionally leads to the coupling of neighboring
$n$-modes.




\section{Shot noise}
The sampling of the density field by a limited number of points leads to a shot
noise component in the power spectrum. To estimate the shot noise, we start
with \citep{Peebles:1973ApJ...185..413P,Feldman+:1994ApJ...426...23F}
\ba
\<n(\vr)\,n(\vr')\>
&=
\bar n(\vr) \, \bar n(\vr') \left[1 + \xi(\vr,\vr')\right]
\vs&\quad
+ \bar n(\vr)\,\delta^D(\vr - \vr')\,,
\\
\<n(\vr)\,n_r(\vr')\>
&=
\alpha^{-1}\,\bar n(\vr) \, \bar n(\vr')\,,
\\
\<n_r(\vr)\,n_r(\vr')\>
&=
\alpha^{-2}\,\bar n(\vr) \, \bar n(\vr')
+ \alpha^{-1}\,\bar n(\vr)\,\delta^D(\vr - \vr')\,.
\ea
The density contrast is given by \cref{eq:density_constrast_discrete_points},
and the ensemble average becomes
\ba
\<\delta^\obs(\vr)\,\delta^\obs(\vr')\>
&=
W(\vr)\,W(\vr')\,\xi(\vr,\vr')
\vs&\quad
+ (1+\alpha)\,\frac{W(\vr)\,\delta^D(\vr'-\vr)}{\bar n}\,,
\ea
where we used \cref{eq:window_definition}.
Therefore, the SFB transform of the shot noise term becomes (see
\cref{eq:sfb_discrete_fourier_pair_b})
\ba
N^\obs
&= 
\frac{1}{\bar n}\,\mathrm{SFB}^2[W(\vr)\delta^D(\vr'-\vr)]
\\
&= \frac{1}{\bar n}\,W_{n\ell m}^{n'\ell'm'}\,,
\label{eq:shotnoise}
\ea
in the limit $\alpha\to0$, and the $W$ matrix is defined in
\cref{eq:delta_mixing_matrix_discrete}. The window-corrected shot noise,
therefore, is, in matrix form, $W^{-1}/\bar n$.

For the pseudo-SFB-power-spectrum estimator the shot noise simplifies significantly.
Averaging over the modes $m=m'$ and assuming $\ell=\ell'$, \cref{eq:shotnoise}
becomes
\ba
\label{eq:Nshot_lnn}
N_{\ell nn'}^{\obs}
&=
\frac{1}{\nbar}
\,\frac{1}{\sqrt{4\pi}}\,
\int\dd r\,r^2
\,g_{n\ell}(r)
\,g_{n'\ell}(r)
\,W_{00}(r)
\,,
\ea
where we used \cref{eq:delta_mixing_matrix_discrete_sht}. \cref{eq:Nshot_lnn}
can be implemented very efficiently.


\section{Pixel window}
The pixel window refers to a distortion of the power spectrum due to binning
galaxies into pixels. In the radial direction, we do not bin the galaxies, see
\cref{eq:delta_nl_theta_phi_exact}, and, therefore, we do not have a radial
pixel window \citep{Leistedt+:2012A&A...540A..60L}.

However, the signal in \cref{eq:sfb_clnnobs} is still affected by the pixel
window from the spherical harmonic transform. We correct this by subtracting
the shot noise from the observed power spectrum, then using the \texttt{pixwin}
function of \texttt{HealPy} to correct for the pixel window. We confirm the
accuracy of this procedure with simulations in \cref{sec:sfb_applications}.







\section{Covariance matrix of power spectrum}
\label{sec:sfb_covariance}
In this section we provide a covariance matrix for the SFB power spectrum.
Several approaches have been used previously.
\citet{Percival+:2004MNRAS.353.1201P,Wang+:2020JCAP...10..022W} trace the
likelihood function either on a grid or using Markov Chain Monte Carlo
techniques. \citet{Wang+:2020JCAP...10..022W}, e.g., use simulations to measure
the covariance matrix from suites of mock catalogues. An analytical approach
for the 3D power spectrum multipoles is presented in
\citet{Wadekar+:2020PhRvD.102l3517W}. In this paper, we get an analytical
estimate for the SFB power spectrum assuming that the density contrast is
Gaussian, and we compare to 100 log-normal simulations. Non-Gaussian terms in
the form of the disconnected trispectrum could be included similarly to
\citet{Taruya+:2021PhRvD.103b3501T,Sugiyama+:2020MNRAS.497.1684S}.

Super-sample variance \citep[e.g.,][]{dePutter+:2012JCAP...04..019D,
Lacasa+:2019A&A...624A..61L, Li+:2018JCAP...02..022L} can have a significant
impact on the covariance matrix. \emph{Beat coupling} is mode mixing due to the
window function with correlation between pairs of non-linear modes and one
large mode. The \emph{local average effect} is due to the large-scale mode
modulating the average number density inside the survey volume. Both these
effects can be treated in the manner of
\cref{sec:sfb_local_average_effect,eq:delta_estimated}.

The covariance matrix on the observed SFB power spectrum is
\ba
\label{eq:Clnnobs_covariance_raw}
&V_{\ell nn'}^{LNN',\obs}
\equiv
\<\hat C^\obs_{\ell nn'} \, \hat C^\obs_{LNN'}\> - C^\obs_{\ell nn'} \, C^\obs_{LNN'}
\vs
&=
\frac{1}{(2\ell+1)(2L+1)}\sum_{mM}\Big[
  \<\delta^{W,A}_{n\ell m}\delta^{W,A}_{NLM}\> \<\delta^{{W,A},*}_{n'\ell m}\delta^{{W,A},*}_{N'LM}\>
\vs&\quad
+\<\delta^{W,A}_{n\ell m}\delta^{{W,A},*}_{N'LM}\> \<\delta^{W,A}_{NLM}\delta^{{W,A},*}_{n'\ell m}\>
\Big],
\ea
where we used Wick's theorem for a Gaussian density contrast. We simplify
\cref{eq:Clnnobs_covariance_raw} in \cref{sec:sfb_covariance_simplification}.
However, an analytical calculation remains computationally expensive.

To get the covariance matrix for the window-corrected power spectrum,
we write the matrix equation
\ba
\label{eq:covariance_matrix_window_deconvolution}
V
&=
\mathcal{N}^{-1}\,V^{\obs}\,\mathcal{N}^{-1,T}\,.
\ea
where $\mathcal{N}$ is the bandpower-binned window coupling matrix given in
\cref{eq:sfb_N_mixing}, and the binning of the covariance matrix is implied.

A reasonably precise estimate can be obtained by counting modes and assuming
the covariance matrix is diagonal. That is,
\ba
\label{eq:sfb_covariance_modecounting}
V_{\ell nn'}^{LNN'}
&\simeq
\frac{\delta^K_{\ell L}}{N_\mathrm{modes}}
\left[
  C_{\ell nN}^\mathrm{binned}
  C_{L n'N'}^\mathrm{binned}
  +
  C_{\ell nN'}^\mathrm{binned}
  C_{L n'N}^\mathrm{binned}
\right],
\ea
where the power spectrum includes the shot noise, $C_{\ell nn'}^\mathrm{binned} =
C^\mathrm{signal}_{\ell nn'}+N^\mathrm{shot}_{\ell nn'}$, and
\ba
\label{eq:sfb_Nmodes}
N_\mathrm{modes} &= f_\mathrm{vol} \,(2\ell+1)\,\Delta\ell\,\Delta n
\,,
\ea
where $\Delta\ell$ and $\Delta n$ are the bin widths for modes $k_{n\ell}$, and
$f_\mathrm{vol}$ is the fraction of the SFB transform-volume that is occupied
by the survey, defined by
\ba
f_\mathrm{vol}
&\equiv
\frac{1}{V_\mathrm{SFB}}\int\dd^3\vr\,\tau\!\left[W(\vr) - W_\mathrm{threshold}\right],
\ea
where $\tau(x)$ is a step function and $W_\mathrm{threshold}$ is a threshold of
the window function.

The shot noise takes into account the variation of the number density across
the survey, and it enters in \cref{eq:sfb_covariance_modecounting} as part of
the power spectrum. The incomplete volume coverage enters as a reduction in the
number of modes, and it is needed for the stability of the window-deconvolution
when there are large unobserved regions in the SFB volume.

In \cref{fig:covariance_matrix} we show the covariance matrices for a set of
simulations that contain only shot noise (top left) as well as for a set of
simulations with a physical galaxy power spectrum with bias $b=1.5$ at
effective redshift $z_\mathrm{eff}=2$ (top right). In the figure we also show
the analytical result from \cref{eq:Clnnobs_covariance_raw} (bottom panels).

The colorbar in the figure is nonlinear. As a result, small elements appear
amplified. To provide a more useful comparison, we introduce the difference
between two covariance matrices, scaled to the center diagonal. That is, we
introduce the relative difference
\ba
\label{eq:drho}
\Delta\rho_{ij} &= \frac{C^A_{ij} - C^B_{ij}}{\sqrt{C^B_{ii}\,C^B_{jj}}}\,,
\ea
and we choose $C^B$ to refer to the analytic result. $\Delta\rho$ does not
suffer from amplification of small differences far from the diagonal.

Therefore, in \cref{fig:covariance_matrix_drho} we show the relative difference
between the covariance matrix as obtained from simulations and the analytical
result. However, in the figure we remove the largest mode, since we have not
included the local average effect in the analytical calculation. All other
modes are statistically essentially equal between simulation result and
analytics.

To show this more clearly, we present \cref{fig:covariance_matrix_diagonals},
where we compare the main diagonal and the $\ell=L+1$ diagonal of the
covariance matrices using the same statistic \cref{eq:drho}. Within the
noise, we find good agreement between simulations and analytical result.









%\acknowledgments


\bibliography{../references}


\appendix


\section{Useful formulae}
\label{app:sfb_useful_formulae}
For any function $f(\vk)$
\ba
&\int k^2\dd{k}\,\dd^2\khat\,\delta^D(\vk-\vk')\,f(\vk)
\vs
&= \int\dd{k}\,\delta^D(k-k')\int\dd^2\khat\,\delta^D(\khat-\khat')\,f(\vk)\,.
\ea
Therefore,
\ba
\label{eq:dirac3D}
\delta^D(\vk-\vk')
&= k^{-2} \, \delta^D(k-k')\,\delta^D(\khat-\khat')\,.
\ea
Furthermore,
\ba
\frac{1}{r}\,\delta^D\!\(\frac{1}{r} - \frac{1}{r_0}\)
&= r\,\delta^D\!\(r - r_0\)\,.
\ea




Spherical Bessel functions and spherical harmonics satisfy orthogonality
relations
\ba
\label{eq:jljlDelta}
\delta^D(k-k')
&= \frac{2kk'}{\pi}\int_0^\infty\dd{r}\,r^2\,j_\ell(kr)\,j_\ell(k'r)\,, \\
\label{eq:YlmYlmDelta}
\delta^K_{\ell\ell'}\delta^K_{mm'}
&= \int\dd{\Omega}_{\rhat}\,Y_{\ell m}(\rhat)\,Y^*_{\ell'm'}(\rhat)\,.
\ea
Spherical harmonics can be expressed in terms of a complex exponential and real
associated Legendre functions $\mathrm{P}_\ell^m(x)$ as
\ba
\label{eq:spherical_harmonics}
Y_{\ell m}(\rhat)
&=
e^{im\phi}
\(\frac{(\ell - m)! (2\ell+1)}{4\pi\(\ell+m\)!}\)^\frac12
\mathrm{P}_\ell^m\!\(\cos\theta\).
\ea
The completeness relation is
\ba
\label{eq:Ylm_completeness}
\sum_{\ell m} Y_{\ell m}(\rhat)\,Y^*_{\ell m}(\rhat')
&= \delta^D(\rhat - \rhat')\,.
\ea
Rayleigh's formula decomposes the plane waves into spherical Bessels and
spherical harmonics,
\ba
\label{eq:rayleigh}
e^{i\vq\cdot\vr} &=
4\pi\sum_{\ell',m'} i^{\ell'} j_{\ell'}(qr)\,
Y^*_{\ell'm'}(\qhat)\,Y_{\ell'm'}(\rhat)\,.
\ea
Legendre polynomials can be expressed as a sum over spherical harmonics as
\ba
\label{eq:legendre_spherical_harmonics}
\mathcal{P}_\ell(\khat\cdot\rhat)
&= \frac{4\pi}{2\ell+1}\sum_m Y_{\ell m}(\khat)\,Y_{\ell m}^*(\rhat)\,.
\ea
Flipping the sign of the component angular momentum or the direction of the
argument to spherical harmonics gives
\ba
\label{eq:Ylm_conjugate}
Y^*_{\ell m}(\rhat)
&= (-1)^m Y_{\ell,-m}(\rhat)
\,,\\
\label{eq:Ylm_parity}
Y_{\ell m}(-\rhat)
&= (-1)^\ell Y_{\ell,m}(\rhat)\,.
\ea
The Gaunt factor is
\ba
\label{eq:gaunt_factor}
\mathcal{G}^{\ell L L_1}_{mMM_1}
&=
\int\dd^2\rhat
\,Y_{\ell m}(\rhat)
\,Y_{LM}(\rhat)
\,Y_{L_1M_1}(\rhat)\,,
\ea
and it can be expressed in terms of Wigner-$3j$ symbols,
\ba
\mathcal{G}^{\ell L L_1}_{mMM_1}
&=
\(\frac{(2\ell+1)(2L+1)(2L_1+1)}{4\pi}\)^\frac12
\begin{pmatrix}
  \ell & L & L_1 \\
  0 & 0 & 0
\end{pmatrix}
\vs&\quad\times
\begin{pmatrix}
  \ell & L & L_1 \\
  m & M & M_1
\end{pmatrix}\,.
\label{eq:gaunt_3j}
\ea
The Wigner $3j$ symbols obey an orthogonality relation
\ba
\sum_{mM}
\begin{pmatrix}
  \ell & L & L_1 \\
  m & M & M_1
\end{pmatrix}
\begin{pmatrix}
  \ell & L & L_2 \\
  m & M & M_2
\end{pmatrix}
&=
\frac{\delta^K_{L_1L_2}
\delta^K_{M_1M_2}
\delta^T(\ell,L,L_1)}
{2L_1+1}\,,
\label{eq:3j_orthogonality}
\ea
where $\delta^T(\ell,L,L_1)$ enforces the triangle relation that is also obeyed
by the 3$j$-symbols.
For the Gaunt factor that implies
\begin{widetext}
\ba
\frac{1}{(2\ell+1)(2L+1)}
\sum_{mM}
\mathcal{G}^{\ell L L_1}_{mMM_1}
\mathcal{G}^{\ell L L_2}_{mMM_2}
&=
\frac{
	\delta^K_{L_1L_2}
	\delta^K_{M_1M_2}
	\delta^T(\ell,L,L_1)
}{4\pi}
\,\begin{pmatrix}
  \ell & L & L_1 \\
  0 & 0 & 0
\end{pmatrix}^2
\,,
\label{eq:gaunt_orthogonality}
\ea
\end{widetext}
That is, the Gaunt factor is only nonzero when
\ba
\label{eq:triangle_mmm}
m + M + M_1 = 0\,,
\\
\label{eq:triangle_lll}
|\ell - L| \leq L_1 \leq \ell + L\,.
\ea
Assuming the triangle condition is satisfied, for even $J=\ell+L+L_1$ we have
\ba
\begin{pmatrix}
  \ell & L & L_1 \\
  0 & 0 & 0
\end{pmatrix}
&=
(-1)^{\frac12 J}
\(\frac{(J-2\ell)!(J-2L)!(J-2L_1)!}{(J+1)!}\)^\frac12
\vs&\quad\times
\frac{\(\frac12J\)!}{\(\frac12J-\ell\)!\(\frac12J-L\)!\(\frac12J-L_1\)!}
\,,
\ea
for odd $J=\ell+L+L_1$, those $3j$'s vanish when $m=M=M_1=0$.












\end{document}



